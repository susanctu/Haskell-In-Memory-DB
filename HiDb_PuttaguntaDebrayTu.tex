\title{HiDb: A Haskell In-Memory Relational Database}
\author{
        \textsc{Rohan Puttagunta}
            \qquad
        \textsc{Arun Debray}
            \qquad
        \textsc{Susan Tu}
        \mbox{}\\ %
        \\
        CS240H\\
        \mbox{}\\ %
        \normalsize
            \texttt{rohanp}
        \textbar{}
            \texttt{adebray}
        \textbar{}
            \texttt{sctu}
        \normalsize
            \texttt{@stanford.edu}
}
\date{\today}
\documentclass[10.75pt]{article}
%\documentclass{acmconf}
%\usepackage[pdftex]{graphicx}
\usepackage{listings}
%\usepackage[paper=a4paper,dvips,top=2cm,left=1.5cm,right=1.5cm,
%    foot=3cm,bottom=3cm]{geometry}
\usepackage[margin=2cm]{geometry}
\usepackage{float}
\usepackage{multicol}
\usepackage[pdftex]{graphicx}
\usepackage[english]{babel}
\usepackage{sidecap}
\usepackage[font=small,labelfont=bf]{caption}
\usepackage[raggedright]{titlesec}

%----------------------------------------------------------

\makeatletter
\newenvironment{tablehere}
  {\def\@captype{table}}
  {}

\newenvironment{figurehere}
  {\def\@captype{figure}}
  {}
\makeatother

\begin{document}

\maketitle

\begin{abstract}
We describe our experience implementing an in-memory relational database in Haskell that supports the standard CRUD (create, read, update, delete) operations while providing the requisite ACID (atomicity, consistency, isolation, durability) guarantees. We rely on Haskell's STM module to provide atomicity and isolation. We use a combination of STM, Haskell's type system, dynamic type-checking in order to enforce consistency. We implement undo-redo logging and eventual disk writes to provide durability. We also provide a Haskell library which clients can use to connect to and send transactions to the database. We found that while the STM module greatly eased the implementation of transactions, the lack of support for de-serializing data into a dynamic type was not ideal. 
 
\end{abstract}

\vspace{5mm}
\begin{multicols}{2}

\section{Introduction} 

\section{Database Operations}
\subsection{Supported Syntax}
The database operations that we support are \texttt{CREATE TABLE}, \texttt{DROP TABLE}, \texttt{ALTER TABLE}, \texttt{SELECT}, \texttt{INSERT}, \texttt{SHOW TABLES}, \texttt{UPDATE}, and \texttt{DELETE}.  We support the following syntax for specifying these operations:
TODO ARUN \\\\
\subsection{Implementation}
Each operation is implemented as a Haskell function.  Operations which make changes to the database, should they succeed, should return lists of \texttt{LogOperation}s, where the \texttt{LogOperation} datatype represents log entries and we use different constructors for different types of entries (see table x). Since we cannot perform IO from within STM, the calling function is responsible for actually writing these \texttt{LogOperation}s, which are instances of \texttt{Read} and \texttt{Show} to allow for easy serializability and de-serializability, to the in-memory log. If the operation was malformed (for example, the referenced table does not exist, or the user failed to specify the value for a column for which there is no default value), then we return some error message to the user. For operations such as \texttt{SELECT} that do not modify the database, we return a string that can be written back to the user. \\\\
In our Haskell implementation of \texttt{SELECT}, we choose to return a \texttt{Table} rather than a \texttt{String} because while we did not implement this functionality in this verison of HiDb, in theory we perform futher operations involving the returned table. We also chose to make use of a \texttt{Row} datatype, which is a wrapper around a \texttt{Fieldname -> Maybe Element} function. This allows us to use functions of the type \texttt{Row -> STM(Bool)} as the condition for whether a row of a table should be deleted or updated. It also allows us to express an update as \texttt{Row -> Row}. 

\section{Data Structures}
\setcounter{subsection}{-1} %this makes it so that preprocessing is 2.0 so that 2.1 and 3.1 both refer to forward selection

\subsection{Concurrency} 
\noindent\begin{minipage}{.45\textwidth}
\begin{lstlisting}[caption=code 1,frame=tlrb, breaklines=true]
data Table 
  = Table { rowCounter :: Int 
               , primaryKey :: Maybe Fieldname 
               , table :: Map Fieldname Column}

data Column 
  = Column { default_val :: Maybe Element
                   , col_type :: TypeRep
                   , column :: TVar(Map RowHash (TVar Element))
                   } -- first element is default value

data Element = forall a. (Show a, Ord a, Eq a, Read a, Typeable a) => 
  Element (Maybe a) -- Nothing here means that it's null
\end{lstlisting}
\end{minipage}\hfill

\section{Parsing}

\section{Durability}

\section{Summary}


\end{multicols}

\begin{thebibliography}{11}

\bibitem{harris} http://research.microsoft.com/pubs/67418/2005-ppopp-composable.pdf

\end{thebibliography}
\end{document}